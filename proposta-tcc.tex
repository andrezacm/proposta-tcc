\documentclass[conference]{IEEEtran}

\usepackage[utf8]{inputenc}
\usepackage[brazil]{babel}

% *** GRAPHICS RELATED PACKAGES ***
%
\ifCLASSINFOpdf
  % \usepackage[pdftex]{graphicx}
  % declare the path(s) where your graphic files are
  % \graphicspath{{../pdf/}{../jpeg/}}
  % and their extensions so you won't have to specify these with
  % every instance of \includegraphics
  % \DeclareGraphicsExtensions{.pdf,.jpeg,.png}
\else
  % or other class option (dvipsone, dvipdf, if not using dvips). graphicx
  % will default to the driver specified in the system graphics.cfg if no
  % driver is specified.
  % \usepackage[dvips]{graphicx}
  % declare the path(s) where your graphic files are
  % \graphicspath{{../eps/}}
  % and their extensions so you won't have to specify these with
  % every instance of \includegraphics
  % \DeclareGraphicsExtensions{.eps}
\fi

% correct bad hyphenation here
\hyphenation{op-tical net-works semi-conduc-tor}


\begin{document}

\title{Titulo}

\author{\IEEEauthorblockN{Michael Shell}
\IEEEauthorblockA{School of Electrical and\\Computer Engineering\\
Georgia Institute of Technology\\
Atlanta, Georgia 30332--0250\\
Email: http://www.michaelshell.org/contact.html}
\and
\IEEEauthorblockN{Homer Simpson}
\IEEEauthorblockA{Twentieth Century Fox\\
Springfield, USA\\
Email: homer@thesimpsons.com}
\and
\IEEEauthorblockN{James Kirk\\ and Montgomery Scott}
\IEEEauthorblockA{Starfleet Academy\\
San Francisco, California 96678-2391\\
Telephone: (800) 555--1212\\
Fax: (888) 555--1212}}

\maketitle


\begin{abstract}
The abstract goes here.
\end{abstract}

\IEEEpeerreviewmaketitle



\section{Introdução}

Um ecossistema de software consiste em uma plataforma de software. Um conjunto de desenvolvedores internos e externos e uma comunidade de especialistas no domínio em serviço da comunidade de usuários que compõem elementos de soluções relevantes para satisfazer suas necessidades [1]. Linux, NojeJS, Windows Kinect e Firefox são exemplos de ecossistemas de software em diferentes contextos. 

Ecossistema de software mobile refere-se as plataformas que dão suporte ao desenvolvimento de aplicativos para dispositivos móveis, como Android, iOS, Windows Phone e Firefox OS. Um ecossistema de software mobile, assim como outros ecossistemas de software, é formado pela colaboração entre diferentes grupos, como empresas, engenheiros de software, programadores, clientes e entidades governamentais [3][37]. Visando o crescimento de seus ecossistemas, as plataformas móveis utilizam diferentes estratégias para motivar a colaboração entre esses grupos [39]. Uma das estratégias utilizadas é o desenvolvimento de um marketplace que possibilite aos desenvolvedores a disponibilização de aplicativos para os usuários finais de smartphones. De modo que os usuários finais obtem o aplicativo através de um App Store e a plataforma repassa uma parcela de 70\% do dinheiro da venda para os desenvolvedores, no caso de aplicativos pagos. Somente na App Store da Apple, até junho de 2013, estavam disponíveis para download cerca de 900000 aplicativos, dos quais 93% são instalados em dispositivos móveis todos os meses. 
http://www.theverge.com/2013/6/10/4412918/apple-stats-update-wwdc-2013  

Aplicativos móveis estão mudando o modo como as pessoas experimentam a computação e como utilizam seus telefones celulares [41]. Oferecer aplicativos variados e de qualidade em uma App Store pode ser a diferença para uma plataforma atingir o crescimento constante e conseguir uma grande parcela de consumidores no mercado de smartphones. Pensando nisso, é de suma importancia que a plataforma ofereça motivações para que os desenvolvedores de software adotem sua tecnologia e disponibilizem mais e melhores aplicativos em suas App Stores. Essas motivações podem estar relacionadas a aspectos econômicos, como ter uma grande parcela do mercado, técnicos, como oferecer ferramentas de qualidade, e sociais, como identificação com a filosofia da plataforma.

Diante da variedade de plataformas mobile e da dificuldade do desenvolvimento cross-platform [42], os desenvolvedores tendem a escolher uma plataforma para a qual vão investir seus esforços no desenvolvimento de aplicativos. Diversos fatores podem influenciar nesta escolha, como retorno financeiro, popularidade, qualidade de SDK, boa documentação, identificação e simpatia com a comunidade, qualidade de ferramentas de teste e simulação, transparência, entre outros. 

Na última década, observamos muitos avanços na tecnologia de smartphones. Isto tem feito com que surjam diversas pesquisas na área buscando investigar diversos aspectos computacionais, economicos e sociais. Recentemente, diversos trabalhos tem sido publicados no âmbito de ecossistemas e plataformas móveis. Alguns trabalhos tem focado nos aspéctos técnicos, dificuldades e desafios enfrentados por desenvolvedores de software mobile [43]. Pesquisas que envolvem uma análise entre as plataformas Android e iOS de diversos aspectos. Como Tilson et al [44] que analisa os fatores que influenciam na evolução dessas plataformas e ou Goadrich et al [45] que comparam as duas plataformas com objetivo de saber qual delas deve ser ensinada nas universidades. Outros trabalhos buscam entender os critérios de escolha que influenciam os desenvolvedores na decisão entre diferentes plataformas [40]. Nosso trabalho tem como objetivo investigar as motivações dos desenvolvedores ao adotar uma plataforma, focando nos aspectos sociais, como a influencia de amigos desenvolvedores, e de abertura da plataforma, como transparência e código aberto.


\section{Conclusion}
The conclusion goes here.

\section{Revisão Bibliográfica}

Podemos encontrar diferentes definições de ecossistemas de software na literatura. Lungu et al. [31] definem ecosistema de software como: “A software ecosystem is a collection of software projects which are developed and evolve together in the same environment.”. Enquanto que Bosch e Bosch-Sijtsema [1] ampliam essa definição para: “a software ecosystem consists of a software platform, a set of internal and external developers and a community of domain experts in service to a community of users that compose relevant solution elements to satisfy their needs”.  

O termo “ecossistemas de software” tem como base o termo “ecossistemas de negócio” que é derivado do termo “ecossistemas biológicos”. Sendo assim, o termo “ecossistemas biológicos” é primitivo do termo “ecossistemas de software”. Sabendo disso, podemos observar na literatura algumas metáforas entre esses termos como, por exemplo, o jaguar de Iansiti and Levien [22]. Esse animal é conhecido por alimentar-se de, aproximadamente, 85 espécies de animais diferentes. Apesar de desempenhar um grande papel ajudando no controle de populações, o jaguar é apenas um pequeno elemento dentro do ecossistema em que está inserido. Podemos usar a história do jaguar de Iansiti e Levien para compreender o papel de um prorietário de plataforma. Por exemplo, a empresa Google -- proprietária/governante da plataforma Android -- que apesar de ser um gigante do mercado tecnológico é apenas uma parte do ecossistema de software móvel. Outras metáforas entre ecossistemas de software e ecossistemas biológicos podem ser vistas na literatura. Dhungana et al. [23] ressaltam algumas, como: quantidade finita de recursos; a colaboração e a competição como elementos fundamentais de interação, progresso e equilíbrio nas relações. 

Moore [24] define ecossistemas de negócio como uma comunidade econômica apoiada por uma fundação: organizações e indivíduos interagindo como organismos do mundo de negócios. Alguns dos papéis exercidos pelos membros desta comunidades incluem: fornecedores, produtores, competidores, entre outros. Observamos que esses papéis são semelhantes aos grupos presentes em um ecossistema de software. Moore[24] também diz que durante o tempo, esses membros aprendem a evoluir em conjunto e tendem a alinhar-se em torno de uma ou mais empresas centrais. Em um ecossistema de software, essas empresas centrais podem ser representadas pelos platform leaders ou governantes definidos por Gawer and Cusumano [25].

Durante a análise de 64 trabalhos encontrados na literatura sobre ecossistemas de software, Manikas e Hansen [38] criaram uma divisão entre os ecossistemas de software considerando suas principais características arquiteturais: i) ecossistemas de engenharia de software, este grupo tem como carcterística principal o uso ou a adaptação de práticas de engenharia de software dentro do seu contexto; ii) ecossistemas de negócio e de gestão, este grupo reune os ecossistemas que tem como foco as ações organizacionais e sua perspectiva principal é voltada ao negócio. Os ecossistemas dessa categoria tem uma gerencia organizacional e membros que são responsáveis pelo acompanhamento operacional e de decisão, além de disponibilizarem documentos sobre negócio, organização e aspectos da gestão do ecossistema; iii) ecossistemas de relacionamentos, essa categoria reune os ecossistemas que dão ênfase aos aspectos sociais. Esses aspectos sociais podem ser a comunidade formada em volta da plataforma, as redes sociais, ou um conjunto de atores e regras de comicação e interação de uma plataforma.

Os aspectos sociais de um ecossistema de software são formados por diferentes atores, como empresas, engenheiros de software, programadores, clientes, entidades governamentais, entre outros, que agem em colaboração. Jansen et al [3], assim como Scacchi e Alspaugh [37], afirmam que os principais atores são os produtores (desenvolvedores de softwares), integradores (fornecedores de software independentes, governo, consultores de integração de sistemas, por exemplo) e clientes (consumidores). Os ecossistemas, de uma forma geral, são governados e dirigidos por uma ou mais partes que lucram quando o ecossistema prospera. Geralmente, esses governantes controlam a tecnologia que serve como base para o ecossistema.

As plataformas móveis podem ter diferentes estratégias para a relação com seus atores em seus ecossistemas. Essas plataformas tem diferentes estágios em seu modelo de negócio. Cada estágio contém diferentes estratégias e dinâmicas. Todas as estaratégias incluem desafios competitivos e colaborativos para motivar um maior compromisso dos atores, bem como a evolução e o crescimento de seus ecossistemas. Como afirma o estudo feito por Karhu et al. [39]. 

Assim como em outros tipos de ecossistemas, os ecossistemas de software serão mais resistentes se a comunidade relacionada for diversificada, como é discutido por Mens et al. [32]. Essa diversidade pode estar em diferentes níveis, como: participantes com especialização em linguagens de programação, atividades de depuração, testes, documentação e veículos de comunicação. No caso de ecossistemas baseados em relações, sejam elas tecnológicas ou humanas, existem diferentes tipos de fatores que influenciam nessas relações que podem estar relacionados a tecnologia, economia, comunicação, aprendizado, experiência e preferências. Esses fatores de influência necessitam estar em equilíbrio, para que haja evolução e crescimento da plataforma. 

Segundo Draxler e Gunnar [33], tendo em vista o crescimento e a variabilidade da comunidade visando uma evolução consistente como experimentado pela plataforma de desenvolvimento Eclipse, os recursos devem ser prioritariamente flexíveis, apesar de serem finitos, e minimamente extensíveis. Draxler e Gunnar [33] ainda mencionam que o desenvolvimento de modo transparente contribui para a confiança dos desenvolvedores e a importancia de estar atento aos feedbacks dos usuários entre a concepção e a distribuição do software. Esses aspectos favorecem o desenvolvimento mútuo, como definido por Andersen e Mørch [34]. Essas características  mantem os desenvolvedores cientes dos acontecimentos e motivados a participar de forma mais engajada. No contexto social, Mens et al. [32] defende que a realização de uma boa comunicação é um fator preponderante para o sucesso de qualquer projeto, principalmente os projetos que são desenvolvidos geograficamente distantes. Assim, ecossistemas que oferecem ferramentas que facilitam a difusão de informação como controle de versão, rastreamento de erros, listas de discussão e fóruns especializados,  tem uma maior capacidade de otimizar o processo de desenvolvimento e melhorar a qualidade dos produtos gerados.

De acordo com o “The Onion Model”, Jergensen et al. [35] afirmam que o envolvimento dos participantes é crescente de modo que os mais novos começam pelas camadas mais externas com menos responsabilidades e, progressiva e lentamente, atingem camadas de maiores responsabilidades. Acreditamos que este modelo pode ser extensível a qualquer ecossistema.

Trabalhos recentes vem explorando o campo de dificuldades e desafios do desenvolvimento de software para aplicativos móveis. Joorabchi et al [43] dão uma visão geral dos desafios atuais enfrentados pelos desenvolvedores de aplicativos mobile na prática, tais como o desenvolvimento de aplicativos em múltiplas plataformas, a falta de monitoramento robusto, ferramentas de análise e teste, e emuladores que são lentos ou estão em falta com as funcionalidades dos dispositivos móveis.

Os resultados do trabalho de Koch e Kerschbaum [40] indicam que as principais motivações para os desenvolvedores ao escolher uma plataforma de desenvolvimento são “experience of fun” e “intellectual stimulation” durante o aprendizado e o processo de desenvolvimento. 


% references section

% can use a bibliography generated by BibTeX as a .bbl file
% BibTeX documentation can be easily obtained at:
% http://www.ctan.org/tex-archive/biblio/bibtex/contrib/doc/
% The IEEEtran BibTeX style support page is at:
% http://www.michaelshell.org/tex/ieeetran/bibtex/
%\bibliographystyle{IEEEtran}
% argument is your BibTeX string definitions and bibliography database(s)
%\bibliography{IEEEabrv,../bib/paper}
%
% <OR> manually copy in the resultant .bbl file
% set second argument of \begin to the number of references
% (used to reserve space for the reference number labels box)

\bibliographystyle{IEEEtran}
\bibliography{proposta-tcc}

\end{document}

